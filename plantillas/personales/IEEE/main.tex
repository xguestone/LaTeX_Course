%%=====================================================================================
%%
%%       Filename:  main.tex
%%
%%    Description:  Plantilla para proyecto en formato IEEE
%%
%%        Version:  1.0
%%        Created:  31/10/18
%%       Revision:  none
%%
%%         Author:  Herbert Arias (dayanqwe123@gmail.com), 
%%   Organization:  
%%      Copyright:  Copyright (c) 2018, Herbert Arias
%%
%%          Notes:  
%%
%%=====================================================================================

\documentclass[twocolumns,a4paper]{IEEEtran}
\usepackage{graphicx}
\usepackage{amsmath}
\usepackage{authblk}
\usepackage[spanish]{babel}
\usepackage{blindtext}
\usepackage[utf8]{inputenc}
\usepackage{hyperref} 
\hypersetup{hidelinks}
\usepackage[style=ieee,backend=biber]{biblatex} 
\bibliography{Bibliography.bib}

\title{Protocolo: Propuesta de Formato de Presentación de Taller de Investigación I}
\author[1]{G. M. Chávez-Campos}
\author[2]{H. J. Simpson}
\affil[1]{Departamento de Ingeniería Electrónica, Instituto Tecnológico de Morelia}
\affil[2]{Twenty Century Fox}

\begin{document}
\maketitle %Título
\begin{abstract}
   El resumen debe de ser preferentemente un parrafo conciso, de una longitud
   aproximada de entre 200 a 350 palabras. Se recomienda que el resumen sea
   escrito al final, los autores deben de incluir la relevancia del tópico, que
   aspectos se abordan en el trabajo, cuál es la metodología, tesis, hipótesis
   y que resultados se obtuvieron.
\end{abstract}
\section{Semblanza del problema}
%Información relacionada con el problema a resolver e.g. - Generación de energía en lugares aislados.
En esta sección se recomienda presentar la relevancia del tópico que se aborda
en el trabajo desde un aspecto muy general. De ser necesario el autor debe
definir terminología, conceptos o aspectos un poco más particulares. De las
misma forma debe resaltarse las ventajas y/o aplicaciones del tópico que se
aborda.

Por otro lado, conforme al autor avanza debe indicar cuales son las debilidades
o desventajas del tópico abordado. Enfatizando el problema que se pretende
resolver. Consecuentemente el autor debe especificar que puntos o áreas serán
las que aborde o resuelva en el proyecto que se presenta\cite{webdev:2018:online}.

%\blindtext[1] %Texto dummy
%-------------------------------------
\section{Revisión del estado del arte}
%En esta sección describir los avances recientes para resolver el problema planteado

A lo largo de la presente sección el autor debe presentar que han hecho otros
autores, empresas o investigadores, para intentar resolver el problema
planteado. En ese sentido es importante que el autor haga las citas
bibliográficas necesarias a otros trabajos, datos o información para respaldar
la veracidad y formalidad del trabajo presentado; evitando hacer
plagio\cite{Chavez-Campos2016}\cite{pascual199012}.

Se recomienda ampliamente que dentro de esta sección se presenten esquemas,
figuras, diagramas, patentes y/o procesos de otros autores. En el código de
\LaTeX{}, se muestra como insertar figuras y hacer referencias a éstas, como la
\figurename{} \ref{ref:FiguraA}. 


\begin{figure}[t!]
   \centering
   \includegraphics[width=2.5in]{example-image-a}
   \caption{Información textual de la figura.}\label{ref:FiguraA}
\end{figure}

De ser necesaria la presentación de ecuaciones o modelos desarrollados por
otros autores, esta es la sección adecuada. Un ejemplo del código para escribir
ecuaciones se presenta en \eqref{eq:y}.

\begin{equation}
   y(k)=\sum_{n=1}^{\infty}\frac{X(k)-X(k-1)}{X(k-2)}\label{eq:y}
\end{equation}

\section{Solución propuesta}
Sección donde se describe y presenta la metodología para resolver el problema,
normalmente se presenta un esquema que indique el montaje experimental.

\section{Tesis o hipótesis}
La tesis de un trabajo es la idea que el autor sustenta o propone para resolver
el problema. Utilizando los argumentos expuestos  durante la semblanza del
problema y la revisión del estado del arte.

%Presentar la propuesta del autor

\section{Objetivos}
\subsection{Objetivo general}
Durante la redacción del objetivo general utilice verbos en infinitivo,
tratando de enmarcar éste en el problema a resolver, ejemplo: ``Identificar las
causas de corto circuito en un sistema X, 
a través de  la metodología Y, contribuyendo  a reducir Z''

\subsection{Objetivos específicos}
\begin{itemize}
   \item  Identificar  x, 
   \item Construir y,
   \item Medir w, 
   \item Analizar  z
\end{itemize}
\section{Metodología}
En esta sección se propone el montaje experimental o esquema de conexión que se
usará para resolver el problema. Utilice una imagen que permita identificar las
variables o datos que ayudaran a determinar la veracidad de la hipótesis.  

\section{Plan y presupuesto}
Hacer una tabla o diagrama de las etapas o requisitos para el desarrollo del
proyecto, usualmente se presentan diagramas de Gantt\cite{mycite2016}.

\section{Conclusiones}
Escribir aquí con conclusiones, si a esta altura del trabajo existen. 

\printbibliography
\end{document}







