%%=====================================================================================
%%
%%       Filename:  main.tex
%%
%%    Description:  Plantilla para proyecto en formato IEEE
%%
%%        Version:  1.0
%%        Created:  31/10/18
%%       Revision:  none
%%
%%         Author:  Herbert Arias (dayanqwe123@gmail.com), 
%%   Organization:  
%%      Copyright:  Copyright (c) 2018, Herbert Arias
%%
%%          Notes:  
%%
%%=====================================================================================

\documentclass[a4paper, twocolumns]{IEEEtran}
\usepackage{graphicx}
\usepackage{amsmath}
\usepackage{authblk}
\usepackage[spanish]{babel}
\usepackage{blindtext}
\usepackage[utf8]{inputenc}
\usepackage{url} 
\title{Rutas de aprendizaje de programación y tecnologías del desarrollo Web en la actualidad}
\author[1]{H. B. Arias}
\author[2]{M. E. Cardenas}
\affil[1]{Facultad de Estudios Generales, Universidad Nacional Mayor de San Marcos}
\affil[2]{Introducción a las Ciencias e Ingeniería}

\begin{document}
\maketitle %Título
\begin{abstract}
   En la actualidad la tecnología tiene un avance vertiginoso y esto genera
   mucho desconcierto al enfrentarse con la decisión de empezar el estudio de
   una de sus ramas, y este avance tiene un cambio mucho mayor en el ámbito de
   la informática ya que comunidades enteras de software así como empresas muy
   grandes del rubro están trabajando en el desarrollo de nuevas y mejoradas
   tecnologías que están reemplazando muy rápido a otras tecnologías
   consideradas nuevas y muy usadas años atrás. En este artículo se realiza un
   breve recopilatorio de esas tecnologías su origen y uso en la Programación
   Web, veremos en líneas generales un panorama de cómo un estudiante de
   primeros ciclos puede abarcar desde el inicio, una carrera profesional
   enfocada a la Programación Web y todo los conocimientos extras que implica
   adquirir a lo largo de su estudio universitario.
\end{abstract}
\section{Introducción}
%Información relacionada con el problema a resolver e.g. - Generación de energía en lugares aislados.
Hoy en día casi todo requiere un tipo de programación. Asi pues, ¿qué es?
Programar es básicamente explicarle a tu ordenador que quieres que haga por ti.
Pero citemos lo que piensan sobre esto a grandes programadores que han
revolucionado algun sector con la programación.  Gabe Newell (creador de
Valve): ``Cuando estás programando le estás enseñando a la cosa posiblemente más
estúpida del universo, un ordenador, a hacer algo''.  Mark Zuckerberg (creador
de Facebook): ``Programar es una de las pocas cosas en el mundo que puedes hacer
cuando estás sentado y simplemente crear algo completamente nuevo desde cero''
Drew Houston (creador de Dropbox): ``Realmente no es muy diferente de tocar un
instrumento o practicar un deporte. Empieza siendo algo muy intimidante, pero
terminas por cogerle el truco.'' Chris Bosh (programador científico de la NBA):
``Programar es algo que puede aprenderse. Y sé que puede ser intimidante,
muchas cosas son intimidantes. Pero ya sabes ¿qué no lo es?''

La programación es algo absolutamente necesaria en nuestra época, ya que está
en el centro de los mejores productos de la tierra, el software ya se apoderó
del mundo. Saber código hace a cualquier profesión mejor, ya que si se aprendes
programar te dará una capacidad impresionante de cambiar tu profesión. Las
personas más exitosas, la gente que tiene los proyectos más exitosos grandes y
de crecimiento en Internet son aquellos que tienen la intersección de dos
conocimientos y uno de esos conocimientos necesarios es la programación.

En el colegio hemos aprendido cosas muy complejas, y más para ingresar a la
universidad se requiere tener cierto nivel de conocimientos complicados como
por ejemplo en química, balancear una ecuación por Redox o en física el uso de
ecuaciones para interpretar el movimiento parabólico de los cuerpos con masa y
aceleración, pero aprender física o química incluso nos puede acercar en cierto
momento a querer aprender programación ya que aprender sobre los
semiconductores en química o los circuitos, y teoría de transistores en física
nos acerca a la programación porque tienen mucho que ver. Entender los
fundamentos de la programación es mucho más sencillo que todo eso. Pero ¿por
qué muchos sino la mayoría no aprendemos programación durante el colegio o más
crítico aún durante la universidad?, esto tiene que ver con el Álgebra y
cálculo, y es que con eso saltamos a las matemáticas que casi siempre son
útiles para Ingeniería Civil o para la ingeniería Bioquímica pero no
necesariamente para la Ingeniería de Sistemas, Ingeniería de Software o las
ciencias de la computación; por ejemplo nos enseñan límites, nos enseñan
integrales, nos enseñan a calcular el área bajo la curva, y es algo muy
importante, pero es algo muy denso, es como si pasáramos de aprender a conducir
un automovil automático a aprender a conducir un automóvil del la fórmula uno,
y programar deja de ser prioridad en la vida universitaria de muchos futuros
ingenieros.

Saber programar es importante y ya quedó claro el por qué, pero ¿por qué
aprender programación Web? Es probablemente una pregunta muy importante.
Y la respuesta llega de inmediato cuando pensamos en el internet y en la Web
que son muy importantes para el uso conectado, además de las computadoras, los
smartphone y muchos otros dispositivos inteligentes que usamos, todos tienen un
componente software que tiene que ver con la programación Web, y es que la
programación Web es amplia y muy interesante. Por citar un ejemplo hablemos de
la aplicación Uber, que es una aplicación que ha cambiado completamente el
servicio de transporte en nuestra ciudad, lo estamos viviendo, y usa
tecnologías web que podemos aprender como MySql y PostgreSql, como base de
datos además de lenguajes como Javascript, Python, Node.js, Go, Java, C, C++,
Objective C y Swift. Como vemos, todos estos nombres pueden sonar desde lo más
raros hasta lo más ostentosos pero son nombres de tecnologías a las que no
deberíamos tener miedo y el objetivo de éste artículo es recopilar la
información y mostrar la ruta que podemos seguir para aprender éstas
tecnologías sin morir en el intento.

El objetivo principal de este breve artículo no es solo el de informar qué
tecnologías web son a las que podemos aprender sino el de instar a formular
proyectos desde ahora ya que el camino de la programación Web implica adquirir
conocimientos, pero más importante aplicarlos y en el camino descubrir qué
nuevos conocimientos debemos y podemos adquirir.

Inicialmente el artículo busca motivar a mis compañeros de base a involucrarse
con el desarrollo Web y que conozcan las herramientas y tecnologías así como
recursos de aprendizaje a la que todos podemos acceder ya que Internet es
libre, pero muchas veces no conocemos los recursos y por eso no logramos
encontrarlos y aprovecharlos.

%\blindtext[1] %Texto dummy
%-------------------------------------
\section{Antecedentes}
%En esta sección describir los avances recientes para resolver el problema planteado

A lo largo de la presente sección el autor debe presentar que han hecho otros
autores, empresas o investigadores, para intentar resolver el problema
planteado. En ese sentido es importante que el autor haga las citas
bibliográficas necesarias a otros trabajos, datos o información para respaldar
la veracidad y formalidad del trabajo presentado; evitando hacer
plagio\cite{ChavezCampos2016}\cite{pascual199012}.

Se recomienda ampliamente que dentro de esta sección se presenten esquemas,
figuras, diagramas, patentes y/o procesos de otros autores. En el código de
\LaTeX{}, se muestra como insertar figuras y hacer referencias a éstas, como la
\figurename{} \ref{ref:FiguraA}. 


\begin{figure}[ht]
   \centering
   \includegraphics[width=2.5in]{example-image-a}
   \caption{Información textual de la figura.}\label{ref:FiguraA}
\end{figure}

De ser necesaria la presentación de ecuaciones o modelos desarrollados por
otros autores, esta es la sección adecuada. Un ejemplo del código para escribir
ecuaciones se presenta en \eqref{eq:y}.

\begin{equation}
   y(k)=\sum_{n=1}^{\infty}\frac{X(k)-X(k-1)}{X(k-2)}\label{eq:y}
\end{equation}

\section{Solución propuesta}
Sección donde se describe y presenta la metodología para resolver el problema,
normalmente se presenta un esquema que indique el montaje experimental.

\section{Tesis o hipótesis}
La tesis de un trabajo es la idea que el autor sustenta o propone para resolver
el problema. Utilizando los argumentos expuestos  durante la semblanza del
problema y la revisión del estado del arte.

%Presentar la propuesta del autor

\section{Objetivos}
\subsection{Objetivo general}
Durante la redacción del objetivo general utilice verbos en infinitivo,
tratando de enmarcar éste en el problema a resolver, ejemplo: ``Identificar las
causas de corto circuito en un sistema X, 
a través de  la metodología Y, contribuyendo  a reducir Z''

\subsection{Objetivos específicos}
\begin{itemize}
   \item  Identificar  x, 
   \item Construir y,
   \item Medir w, 
   \item Analizar  z
\end{itemize}
\section{Metodología}
En esta sección se propone el montaje experimental o esquema de conexión que se
usará para resolver el problema. Utilice una imagen que permita identificar las
variables o datos que ayudaran a determinar la veracidad de la hipótesis.  

\section{Plan y presupuesto}
Hacer una tabla o diagrama de las etapas o requisitos para el desarrollo del
proyecto, usualmente se presentan diagramas de Gantt\cite{mycite2016}.

\section{Conclusiones}
Escribir aquí con conclusiones, si a esta altura del trabajo existen. 

\bibliographystyle{ieeetr}
\bibliography{Bibliography}
\end{document}







